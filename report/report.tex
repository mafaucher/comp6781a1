%%%%%%%%%%%%%%%%
% DESCRIPTION: %
%%%%%%%%%%%%%%%%
%
% Template Author: Marc-André Faucher
% Last update: 13-01-13
%
% 
%%%%%%%%%%%%%%
% REMINDERS: %
%%%%%%%%%%%%%%
%
% Special Characters: # $ % & ~ _ ^ \ { }
%
% Enumerate: \begin{enumerate}[(a)] [1.] [CHAPTER 1 - ]
%		\item 1st entry
%		\item 2nd entry ...
%	    \end{enumerate}
%
% Include Graphics:
%\begin{center}
%	\includegraphics[scale=0.6]{figures/pipeline.png}
%	{\bf Fig. 1:} Modules in the University Pipeline
%\end{center}
%
% Include Table:
%\begin{center}
%{\bf Table 1:} Number of Manual Annotations
%\begin{tabular}{|l|l|l|l|l|l|l|}
%\hline
%Document   & Wellington & Surge & Professor & Candidate & American & Total \\
%\hline
%Token      & 776 & 1009  & 1051      & 1509      & 922      & 5267  \\ 
%Position   & 5   & 0     & 6         & 22        & 20       & 53    \\ 
%University & 9   & 32    & 12        & 38        & 24       & 115   \\ 
%Unit       & 0   & 1     & 8         & 44        & 12       & 65    \\ 
%\hline
%\end{tabular}
%\end{center}
%
% Force line break: \\
% 	Additional: \item \hspace*{\fill} \\ 
%
% Force page break: \pagebreak
% 
% Fixed-width text: \begin{verbatim}...{verbatim}
%
%%%%%%%%%%%
% HEADER: %
%%%%%%%%%%%
%
\documentclass[12pt]{article}
\usepackage[utf8x]{inputenc}
\usepackage[margin=1in]{geometry}
\usepackage[pdftex]{graphicx}
\usepackage{cite,hyperref}
\usepackage{enumerate,url,multicol,setspace,graphicx}
\usepackage{amssymb,amsmath}
\usepackage{listings}
\usepackage{fancyhdr}
%
%%%%%%%%%%%%%
% COMMANDS: %
%%%%%%%%%%%%%
%
\newcounter{excounter}
\newenvironment{ex}{\begin{quote}%
		\refstepcounter{excounter}%
		\textbf{Ex. \arabic{excounter}}%
	}{%
	\end{quote}%
}
\newcommand{\tab}{\hspace*{0.5cm}}
%
%%%%%%%%%%
% TITLE: %
%%%%%%%%%%
%
\pagestyle{fancy}
\fancyhf{}
\lhead{Marc-Andr\'e~Faucher~(9614729)}
\chead{COMP~6781}
\rhead{Assignment~1~---~May~20\textsuperscript{th}~2013}
\begin{document}
%
%\title{ {\bf COMP 6781: Statistical NLP \\
%Assignment 1 } }
%\author{Marc-André Faucher}
%\date{\today}
%\begin{document}
%\maketitle
%
%%%%%%%%%%%%%
% DOCUMENT: %
%%%%%%%%%%%%%
%

\section*{Question 1: Linguistic Essentials}

\subsection*{Exercise: 3.1}

The/\emph{Det} lemon/\emph{Noun} is/\emph{Verb} an/\emph{Det}
essential/\emph{Adj} cooking/\emph{Adj} ingredient/\emph{Noun}. Its/\emph{Pro}
sharply/\emph{Adv} fragrant/\emph{Adj} juice/\emph{Noun} and/\emph{Con}
tangy/\emph{Adj} rind/\emph{Noun} is/\emph{Verb} added/\emph{Verb}
to/\emph{Prep} sweet/\emph{Adj} and/\emph{Con} savory/\emph{Adj}
dishes/\emph{Noun} in/\emph{Prep} every/\emph{Det} cuisine/\emph{Noun}. \\

\footnotesize{Note: The following abbreviations where used:
	\emph{Det}:~\emph{Determiner}, \emph{Adj}:~\emph{Adjective},
	\emph{Adv}:~\emph{Adverb}, \linebreak \emph{Pro}:~\emph{Pronoun},
	\emph{Con}:~\emph{Conjunction}, \emph{Prep}:~\emph{Preposition}.}
	\normalsize{}

\subsection*{Excercise: 3.3}

The subject is \emph{He}, the direct object is \emph{apple pie} and the
indirect object is \emph{her}.

\subsection*{Excercise: 3.4}

In the first sentence, Mary is defending another person marked by the
third person pronoun \emph{her}, while in the second sentence Mary is
defending \emph{herself}, that is, she is defending Mary.

\subsection*{Excercise: 3.9}

Although they can be identical tokens, particles are distinguished from
prepositions because they form a collocation with the preceding verb
which modifies the sense of the verb phrase it is part of. Prepositions
are the head of a prepositional phrase, which can usually be removed
without dramatically changing the semantics of the sentence.

\begin{enumerate}[a.]
	\item Preposition (\emph{in London} is a prepositional phrase).
	\item Particle (\emph{move in} is a semantic unit meaning to change residence).
	\item Particle (\emph{puts in} is a semantic unit meaning spending time or energy).
	\item Preposition (\emph{in the wrong folder} is a prepositional phrase).
\end{enumerate}

\subsection*{Excercise: 3.12}

\begin{enumerate}[a.]
	\item Mary saw the man \emph{with the telescope}.

The prepositional phrase could attach to the verb \emph{saw} (Mary saw the man
through a telescope) or to the \emph{the man} (Mary saw a man carrying a
telescope).

	\item The company experienced growth in classified advertising
			\emph{and preprinted inserts}.

The conjunction could attach to the noun phrase (\emph{classified advertising}
and \emph{preprinted inserts}), in this case \emph{preprinted} is an adjective
much like \emph{classified}. The conjunction could also attach to the verb
phrase (\emph{experienced growth} and \emph{preprinted inserts}), this would
imply \emph{preprinted} is a verb, like \emph{experienced}.
\end{enumerate}

\pagebreak

\section*{Question 2: Collocations}

\subsection*{Introduction}

Finding collocations is an important task in natural language processing (NLP).

\subsection*{Program}

The program used to perform the experiments was written in Python, using the
Natural Language Toolkit (NLTK)\cite{nltk}.

\subsection*{Experiment}
\subsection*{Results}
\subsection*{Future Work}

%1. List the 50 most frequent bigrams in your corpus along with their frequency.
%2. Use a part-of-speech tagger of your choice (see http://www-nlp.stanford.edu/links/statnlp.html#Taggers for
%  free taggers) and develop tag patterns to filter your bigrams. Show the best collocations you find this way.
%3. Develop and experiment with 2 other methods discussed in class to find collocations. Analyze your results
%  and compare them to those you had in step 2 above.
%	
%Write a report (~4 pages) to describe your experiments. Your report must describe:
%
%- The program:
%- Briefly describe your code (choice of language, data structures, ...)
%- Indicate the instructions necessary to run your code (files, commands, ...)
%
%- The experiments:
%- Describe any assumptions that you made (definition of a word)
%- clearly indicate the part-of-speech tagger that you used, and the tag patterns you developed
%- Describe your corpora briefly (size, source, ...)
%- Describe the 2 methods you chose to implement (do not re-explain the theory, but just the parameters
%you used; window size, ...)
%
%- The results:
%- Analyse your results. Explain what went right, and what went wrong. For example, which method
%seemed to give the best results? Do the methods perform the same way on both corpora?
%- Show some examples of the “best/most interesting” collocations that you extracted and show some
%examples of the “worst” collocations that you extracted. Analyse these. Why do you think you have
%the results that you have?
%
%- Future Work:
%- Indicate what you could improve if you had the time and energy.
%
%- References
%- Properly indicate all external sources that you used to do your assignment.

\bibliographystyle{plain}
\bibliography{ref}

\end{document}
